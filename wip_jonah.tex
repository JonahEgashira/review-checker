\documentclass[a4paper, twocolumn]{jsarticle}
\usepackage[dvipdfm]{graphicx}

\begin{document}

\title{ベンフォードの法則によるフェイクレビュー検出手法の検討}
\author{
  NECO B2 江頭叙那(jonah)
  \\ 親 ks91 
}
\maketitle

\section{はじめに}

ネットショッピングが普及し,多くの人々がオンラインで買い物
をするようになった.商品のレビューはユーザーが購入を検討する際に大きな要因となるが,
企業が金銭を見返りに偽のレビューを書かせるフェイクレビューが問題になっている.

\section{目的}
ベンフォードの法則を応用した素性を利用して,既存方式よりも
少ない情報から,商品にフェイクレビューが含まれているかどうか
を検出することを目指す.

\section{ベンフォードの法則の概要}
ベンフォードの法則とは,電気料金の請求書,住所の番地,
株価,人口,革の長さ,物理・数学定数などの自然界に現れる
数値集合の各数値における最上位桁の数値の出現確率には偏り
があるという法則である.この分布から大きく離れている場合,
何らかの人手による操作などが行われた可能性がある.\cite{benford}

\section{仮説}
偽のレビューを書くレビュアーは実際に商品を使っておらず,さらに
報酬を得るために短期間で多くのレビューを書く必要があるため,他のレビューを参考に
似た文章を書くケースが多いのではないかと考える.よって,偽のレビューを含んでいない
商品と比べて使用される文字の頻度の分布が異なるのではないかと考えた.


\section{既存方式}
Amazonのフェイクレビューを見抜くサクラチェッカー \cite{sakura}
では,価格・製品,レビュー分布やショップレビューなど
計8項目の情報を,独自のロジックや機械学習を用いて分析
している.

\section{既存方式の課題}
サクラチェッカーではAmazonのレビューしか判定する
ことができず,他サイトのレビュー判定ができない.
また,分析項目が多く他サイトに応用しづらい点が
課題である.

\section{提案方式}
1.サクラチェッカーでサクラ度が20\%以下の商品を
安全な商品,80\%以上の商品を危険な商品とし,
2グループそれぞれ200商品のレビューをSeleniumによって
取得する.\cite{amazon}
\\2.取得したレビューの文字の出現頻度
をカウントし,頻度の最上位桁の割合を計算する.
\\3.安全な商品と危険な商品のグループで,最上位桁
の分布がどのように異なるかを分析する.
\\4.分析結果をもとに商品にフェイクレビューが
含まれているかどうかを判定する

\section{結果}
サクラチェッカーによる安全な商品と危険な商品のレビューを
それぞれ200商品ずつ取得し,出現する文字の頻度の最上位桁を
取得した結果が以下のグラフである.\\


\begin{figure}[htbp]
  \begin{center}
    \includegraphics[width=7cm]{./bad_result.png}
    \caption{危険な商品の文字頻度の分布}
    \includegraphics[width=7cm]{./good_result.png}
    \caption{安全な商品の文字頻度の分布}
  \end{center}
\end{figure}%


1の位において,危険な商品のほうが出現頻度が高く標準偏差も大きかったが,
差は僅かであった.そのため,レビューの文字出現頻度から
フェイクレビューが含まれるかを判定することはできなかった.
また,janomeによる形態素解析によって単語に分解したものの
頻度を分析しても,2つの商品グループに僅かな差しか認められなかった.

\section{考察}
安全な商品と危険な商品の文字の出現頻度の最上位桁の割合
に差がほとんどなかった理由は以下の3つであると考えた.

\begin{enumerate}
  \item フェイクレビュアーが他のレビューを参考に似た文を書いているという仮説が正しくなかった.
  \item 商品あたりのレビューが少ない場合に,頻度が偏ってしまった.
\end{enumerate}


\section{今後の展望}
ベンフォードの法則を用いてレビューのみからフェイクレビューを検出することは
困難であるとわかった.今後は機械学習の学習を進めて様々な観点からのフェイクレビュー検出を
検討していきたい.

\begin{thebibliography}{99}
  \bibitem{benford} ベンフォードの法則 \\ https://ja.wikipedia.org/wiki/ベンフォードの法則/ (参照 2021/1/27)
  \bibitem{sakura} サクラチェッカー \\ https://sakura-checker.jp/ (参照 2021/1/27)
  \bibitem{vaughan} Lee Vaughan, 高島亮祐訳, ``実用的でないPythonプログラミング'', 第1版, 共立出版(2020)
  \bibitem{kurauchi} 蔵内 雄貴 他, ``ベンフォードの法則を応用したbotアカウント検出'', 日本電信電話株式会社 NTT サービスエボリューション研究所(2013)
  \bibitem{amazon} ジコログ ``Amazonのスクレイピング対策を攻略する'' \\ https://self-development.info/amazonのスクレイピング対策を攻略する【selenium最強説】/ (参照 2021/1/27)
  \bibitem{janome} うぇぶのきわみ, IkeSei, ``pythonで日本語の記事に登場する単語の出現数を調べる方法'' \\ https://web-kiwami.com/count-words-article-python.html (参照2021/1/27)
\end{thebibliography}

\end{document}