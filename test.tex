\documentclass[a4paper, twocolumn]{jsarticle}
\begin{document}

\title{ベンフォードの法則によるフェイクレビュー検出手法の検討}
\author{
  NECO B2 江頭叙那(jonah)
  \\ 親 ks91 
}
\maketitle

\section{はじめに}

ネットショッピングが普及し,多くの人々がオンラインで買い物
をするようになった.ユーザーは商品を買う際にレビューを参考に
するが,企業が金銭を見返りに偽のレビューを書かせるフェイクレビュー
が問題になっている.

\section{目的}
ベンフォードの法則を応用した素性を利用して,既存方式よりも
少ない情報から,商品にフェイクレビューが含まれているかどうか
を検出することを目指す.

\section{ベンフォードの法則の概要}
ベンフォードの法則とは,電気料金の請求書,住所の番地,
株価,人口,革の長さ,物理・数学定数などの自然界に現れる
数値集合の各数値における最上位桁の数値の出現確率には偏り
があるという法則である.この分布から大きく離れている場合,
何らかの人手による操作などが行われた可能性がある.

\section{仮説}
偽のレビューを書くレビュアーは実際に商品を使っておらず,さらに
報酬を得るために短期間で多くのレビューを書く必要があり,他のレビューを参考に
書いているケースが多いのではないかと考える.その場合,偽のレビューを含んでいない
商品と比べて使用される文字の頻度の分布が異なるのではないかと考えた.


\section{既存方式}
Amazonのフェイクレビューを見抜くサクラチェッカー
では,価格・製品,レビュー分布やショップレビューなど
計8項目の情報を,独自のロジックや機械学習を用いて分析
している.

\section{課題}
サクラチェッカーではAmazonのレビューしか判定する
ことができず,他ECサイトのレビュー判定ができない.
また,分析項目が多く他サイトに応用しづらい点が
課題である.

\section{提案方式}
1.サクラチェッカーでサクラ度が20\%以下の商品を
安全な商品,80\%以上の商品を危険な商品とし,
2グループそれぞれ100商品のレビューをSeleniumによって
取得する.
\\2.取得したレビューの文字の出現頻度
をカウントし,頻度の最上位桁の割合を計算する.
\\3.安全な商品と危険な商品のグループで,最上位桁
の分布がどのように異なるかを分析する.
\\4.分析結果をもとに商品にフェイクレビューが
含まれているかどうかを判定する

\section{結果}
箇条書きとともに結果


\begin{itemize}
  \item ちゃお
  \item りぼん
  \item なかよし
\end{itemize}

これは番号を振る箇条書きです。

\begin{enumerate}
  \item 富士
  \item 鷹
  \item なすび
\end{enumerate}

\section{考察}


\end{document}